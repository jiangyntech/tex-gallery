\documentclass[preview]{standalone}
\usepackage{tabularx}
\usepackage{leftidx}
\usepackage[flushleft]{threeparttable}
\usepackage{tikz}
\usepackage[left=2.5cm,right=2.5cm,top=3cm,bottom=3cm]{geometry}
\usetikzlibrary{matrix,positioning,calc}
\newcommand\tikzmark[1]{%
\tikz[remember picture]  \coordinate (#1){};%
}

%% For table drawing, refer to https://tex.stackexchange.com/questions/160423/tikz-table-with-mathmode-and-drawing 
% style inspired by table 4 in https://onlinelibrary.wiley.com/doi/10.1046/j.1095-8312.2002.00008.x
\begin{document}
    
\begin{minipage}[h]{.7\linewidth}
\begin{table} 
    \caption{Population variation in hatch success(mean percent) of unfertilized eggs for females from populations sampled in 1997.}
    \begin{tabular*}{\linewidth}{p{0.4\linewidth}cccl}
        \hline  \\[-10pt]
        Population & mean (\%)  &\begin{tabular}[c]{@{}l@{}}Standard\\ deviation\end{tabular}   & Range & N \tikzmark{columnTitle} \\ \hline
        Population name1$^T$ & 10 & 1 & 0-50 & 10 \\ 
        Population name2$^T$ & 10 & 1 & 0-50 & 10 \\ 
        Population name3$^T$ & 10 & 1 & 0-50 & 10 \\ 
        Population name4$^P$ & 10 & 1 & 0-50 & 10 \\ 
        Population name5$^P$ & 10 & 1 & 0-50 & 10 \tikzmark{tableBody} \\ 
        Population name6$^P$ & 10 & 1 & 0-50 & 10 \\ 
        Population name7$^L$ & 10 & 1 & 0-50 & 10 \\ 
        Population name8$^L$ & 10 & 1 & 0-50 & 10 \\ 
        Population name9$^L$ & 10 & 1 & 0-50 & 10 \tikzmark{endNode}\\ 
        \hline 
    \end{tabular*}
    \begin{tablenotes}
    \item $\leftidx{^T}{}$ = \footnotesize{temporary stream},
    $\leftidx{^P}{}$ = \footnotesize{permanent streams}, 
    $\leftidx{^L}{}$ = \footnotesize{lakes}.\tikzmark{footnoteNode}
    \item
    \item 
    \end{tablenotes}
\end{table}
\end{minipage}

% draw an arrow annotation
\tikzset{arrow/.style ={draw, line width=1.5pt,dashed}}
\newcommand{\drawArrowTxt}[2]{
    \node[draw=none,red,anchor=west,text width=3cm] at ($(#1.east)+(0.8cm,0)$) {#2};
    \draw [red,arrow]  ($(#1.east)+(0.8cm,0)$) -- +(-0.5cm,0);
}

\tikz\coordinate (captionNode) at ($(columnTitle)+(0,0.85)$);
\tikz\coordinate (endNode1) at ($(endNode)+(15pt,-5pt)$);
\begin{tikzpicture}[remember picture,overlay,->]
    %\tikz \node[inner sep=0,outer sep=0] (#1){};%
    \drawArrowTxt{captionNode}{Table legend}
    \drawArrowTxt{columnTitle}{Column titles}
    \drawArrowTxt{tableBody}{Table body(data)}
    \drawArrowTxt{endNode1}{\footnotesize{Lines demarcating the different parts of the table}}
    \drawArrowTxt{footnoteNode}{footnodes}
\end{tikzpicture}

\end{document}